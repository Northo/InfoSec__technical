\documentclass[conference]{IEEEtran}
\IEEEoverridecommandlockouts
% The preceding line is only needed to identify funding in the first footnote. If that is unneeded, please comment it out.
%\usepackage{cite}
\usepackage{amsmath,amssymb,amsfonts}
\usepackage{algorithmic}
\usepackage{graphicx}
\usepackage{textcomp}
\usepackage{xcolor}
\usepackage{quoting}
\usepackage{microtype}
\def\BibTeX{{\rm B\kern-.05em{\sc i\kern-.025em b}\kern-.08em
    T\kern-.1667em\lower.7ex\hbox{E}\kern-.125emX}}

%\usepackage{ieeetr}{biblatex}
%\bibliography{sample}

\begin{document}
\title{On post-quantum cryptography}
\title{Encryption in the post-quantum era}
\title{The need for new encryption in the era of quantum computing}

\author{\IEEEauthorblockN{Thorvald Ballestad thorvald.tb@gmail.com}}
%% \author{\IEEEauthorblockN{Thorvald Ballestad thorvald.tb@gmail.com}
%% \IEEEauthorblockA{\textit{Department of Physics, NTNU} -- Trondheim, Norway}
%% }

\maketitle

\begin{abstract}
  The discovery of assymmetric encryption in the early 70s has blessed our society with secure and available encryption of online communication.
  In essence, the simple fact that the factorization problem appears to be very difficult to solve, allows us to use public-key encryption in the key agreement phase of network encryption.
  The advancement of real-world quantum computers threaten to destroy this privilege -- the factorization problem is simple on a quantum computer.
  There is only a matter of time before quantum computers will be powerful enough to break assymetric encryption -- which will destroy the very foundation of online security.
  To combat this, one seeks to introduce post-quantum cryptographic methods.
  Algorithms for encryption that can not simply be broken by quantum computers.
  In this paper, we will explore the need for new encryption in the era of quantum computing.
\end{abstract}

%% \begin{IEEEkeywords}
%% quantum, computer, encryption
%% \end{IEEEkeywords}

\section{Introduction}
At its core, all communication over the internet is essentially like shouting to each other in a crowded environment, open for anyone to hear.
At the infancy of the internet, this was fine, as the communication was only between trusted parties.
Today the situation is rather different, with personal details, photos, banking details, business secrets, and much more being transported over the internet.
We need to verify the identity of the other party, and also ensure that no other parties may read our communication -- we need authentication and encryption.
Authentication is the process of verifying the identity of the parties involved in the communication.
Encryption is the process of obfuscating the information in the communication, so that it is unreadable to anyone except the intended recipient(s).
Most of the secure communication on the internet today is provided by the Transport Security Layer (TSL)\cite{sslpulse.2020}, which provides both authentication and encryption.

\section{A quick introduction to encryption}
In the introduction we mentioned encryption and authentication, both of which are relevant in post-quantum cryptography.
For the sake of brevity however, we focus here on encryption.

There is an inherent issue with encryption over the web -- how does two parties, who have never met in person, establish a secure connection that is obfuscated for anyone else, when the establishment of such communication must happen over open communication?
The answer is asymmetric encryption, also known as public-key encryption
\footnote{The author is at unease calling general asymmetric encryption public-key encryption, as it paints a misleading mental image for some algorithms. The term \emph{asymmetric} will therefore be used throughout.}.

The most well known asymmetric encryption algorithm is RSA, which the reader might have heard of.
There is also the Diffie-Hellman algorithm, with its many variants, which is becoming increasingly more popular, for its many advantages over RSA\cite{cryptographyStallings}.
They do both, however, rely on the same basic principle: asymmetric encryption schemes are methods for establishing a secure connection between parties without requiring a pre-established shared secret.

%% In modern encryption, asymmetric encryption is mainly used as a key-exchange protocol.
%% Symmetric schemes, methods that require a shared secret, are hugely superior in performance -- about 10'000 times faster.

Asymmetric encryption must be seen in contrast to \emph{symmetric} encryption, such as the Advanced Encryption Standard (AES).
In symmetric encryption, the involved parties must have a shared secret before the communication initiates.
The advantage of symmetric encryption over asymmetric encryption is performance -- symmetric schemes are about 10'000 times faster than asymmetric schemes.
For this reason, the bulk of online communication is encrypted using symmetric encryption, while asymmetric schemes are used only for establishing the key needed in the symmetric scheme -- a process known as key exchange.

%% Asymmetric encryption must be seen in contrast to \emph{symmetric} encryption, such as the Advanced Encryption Standard (AES).
%% In symmetric encryption, the involved parties must have a shared secret before the communication initiates.
%% As a trivial example, take the Caesar cipher, where each letter of the alphabet is ``shifted'' some number $n$ places.
%% For example, for $n=3$ the string ``ABC'' would be ``DEF''.
%% This encryption only works if both parties already know that $n=3$, which is the shared secret, or \emph{key}.
%% The aforementioned AES is the de facto standard for asymmetric encryption, used in TLS.
%% One might wonder why a symmetric cipher would be used on the internet, where we have just established that the parties have no shared secret -- a problem solved by \emph{asymmetric} encryption.
%% The answer is simply performance.
%% Asymmetric encryption is slow.
%% About 10'000 times slower than symmetric encryption.
%% The clever solution is then to initiate the communication with asymmetric encryption, such as Diffie-Hellman, and use that to establish some shared secret.
%% That shared secret is then the key for the rest of the communication, which is encrypted with a symmetric method, thus getting the performance benefit of symmetric encryption.
%% The process of using asymmetric encryption to establish a shared key for the rest of the session, is known as \emph{key exchange}.
%% As we will see, it is this process that will be of interest in our discussion on quantum computers.

\subsection{The basic principle of asymmetric encryption}
Both RSA and Diffie-Hellman relies on the fact that there are no known algorithms for solving the hidden subgroup problem quickly.
The hidden subgroup problem is a generalization of the factorization problem.
As a practical example, consider the following written by the mathematician William Jevons in 1874\cite{oldMathGuy}:
\begin{quoting}
Can the reader say what two numbers multiplied together will produce the number 8616460799? I think it unlikely that anyone but myself will ever know.
 
- William Stanley Jevons, 1874
\end{quoting}
The answer is $89681 \times 96079$, which is not difficult to find with modern computers.
However, the concept stands: given two primes a and b, it is simple to compute $N = a \times b$.
However, if a and b are sufficiently large, it is extremely hard to find a and b given only N.
A similar problem exists for discrete logarithms: given numbers $N$ and $a$, find the integer $k$ such that $a^k = N$.
One may show that solving one of the problems, allows you to solve the other.

Both Diffie-Hellman and RSA are, though not terribly complicated, too lengthy to describe in depth here, so a quick outline will suffice.
We will here describe the fundamentals of Diffie-Hellman\cite{diffieHellman}.
Suppose we have two people, Alice and Bob, who wish to establish encrypted communication.
Alice and Bob first generate a number each, a and b, which are kept secret.
They also agree on some number g, which is open for anyone to see.
Alice then computes the number $g^a$, and sends it to Bob.
Bob computes $g^b$, which he sends to Alice.
By the simple fact that $(g^b)^a = g^{ab} = (g^a)^b$, they may now both compute $g^{ab}$, which they use as a symmetric key.
Notice that the only numbers ``available'' to anyone, the ones that are sent in the open, are $g, g^a$ and $g^b$.
As was mentioned earlier, finding $a$ from $g^a$ is difficult, and thus an adversary is unable to find $g^{ba}$.
Of course, note also that $g^a g^b = g^{a+b}\neq g^{ab}$.

%% Firstly, let us describe RSA.
%% Both Alice and Bob will have a pair of keys (numbers), one private and one public.
%% What this means is that the private key is a secret, while the public key is open for anyone to read.
%% The keys are such that something encrypted with one of the keys, may only be decrypted with the other.
%% One party, let us say Alice, will initiate the exchange by generating a new key, that will be the shared secret for the symmetric exchange -- this is the key that she must communicate to Bob, without anyone else being able to read it.
%% She then encrypts in with Bob's public key, and sends the result to Bob.
%% The only way to decrypt the message, is to use Bob's private key, which only Bob knows.
%% They now both have access to the symmetric key that Alice generated!


\section{Why are quantum computers a threat?}
Put quite pedantically, the reason that the factorization problem is hard to solve, is that given $N = a \times b$, with $a,b$ some unknown primes, we know no function $f$ such that $a = f(N)$ (or $b=\tilde{f}(N)$ for that matter).
One must simply guess $a$ and $b$ until we find some numbers that produce $N$ -- a brute force attack. \footnote{There exists techniques for limiting what $a$ and $b$ may be and finding the most efficient order in which to choose. However the principle stands.}
With $N$ being numbers with hundreds of digits, this problem is simply too time consuming to be practically solvable.
%% For the technically inclined, we mention that the best known \emph{classical} method for finding the factors of $N$ is the general number field sieve, which in essence is a clever way to brute force, has a complexity of
%% $\mathcal{O}(e^{1.9 (\log N)^\frac13 (\log\log N)^\frac23})$ with the important part being that we have an exponential function.
%% The best known \emph{quantum} algorithm has a complexity of only $\mathcal{O}((\log N)^2 (\log \log N)(\log \log\log N))$, with the important part being that the leading term here is some power of the logarithm.
On quantum computers there does however exist an efficient method for solving the factorization problem.
That algorithm is known as Shor's algorithm.

We will not in this article give a detailed explanation of neither how Shor's algorithm works nor how a quantum computer works, as that is vastly outside the scope of this article.
A superficial explanation will suffice, however the author is confident that it will be sufficient in order to appreciate the threat of quantum computers.

On a normal computer numbers are represented by \emph{bits} -- something that may take one of two values.
Positional notation with base two allows us to represent any integer using this system.
For example, 3 is 0011 and 9 is 1001, where the two values each bit can take is here 1 or 0.
A quantum computer works in much the same way.
It has qbits, which may have one of two values, 1 or 0.
However, exploiting the nature of quantum mechanics, we may put one quantum bit in several states at the same time, say 1 \emph{and} 0.
Or we may say that it is almost entirely in the state 1, but also a little in the state 0, say 90\% and 10\%.
The important thing to note is that the qbit is \emph{not} in the state 0.5 or 0.9!
It is simply in the state 1 \emph{and} 0, a concept that only exists in quantum mechanics that is not relatable in our classical world.

Given four qbits, we may have the state 1001 \emph{and} 0011 at the same time, or 9 \emph{and} 3 in decimal.
Like classical computers, quantum computers may perform computations on these numbers, for example add one, giving 10 and 4, or 1010 and 0100.
Importantly, the quantum computer did not perform 9+1=10 and \emph{then} 3+1=4.
It performed (9 \emph{and} 3) + 1 = 10 \emph{and} 4!
Quite simplified, Shor's algorithm exploits this by trying all possible factors of $N$ at the same time.
As a mental image, though not correct, it may be helpful to imagine that we have the state 2 and 3 and 5 and 7 and \dots, and then divide $N$ by that state, we get $N$/2 and $N$/3 and $N$/5 and $N$/7 and \dots, and we simply find the number that is also a integer. \footnote{Note that this is \emph{vastly} simplified. Anyone with knowledge of quantum mechanics will immediately recognize the issue of measurement in this example. The actual algorithm exploits a clever measurement giving a periodic state that through a quantum fourier transform may more easily be measured.}

%% Note to self:
%% Shor's algorithm may be run on a classical machine, but one of the steps is slow.

\subsection{The solution}
It is obvious that a post-quantum secure encryption standard has to rely on some different mechanism than the current standards -- the factorization problem.
As is common in cryptography research one tries to find methods that can be shown to reduce to known mathematical problems, for which there exist no known fast algorithm -- just like one did with the reduction of RSA and Diffie-Hellman into the hidden subgroup problem (integer factorization).
In 2016 the National Institute of Science and Technology (NIST) initiated a standardization process for finding new methods of authentication and key exchange for the post-quantum era.
This process has gone through several rounds of discarding some submissions and moving on with the most promising.
As late as July 2020 the third round was announced, making this a very relevant topic.
In the third round seven submissions remain, four schemes for key exchange and three for authentication.

% Maybe interesting to compare size of keys and efficiency to RSA and Diffie-Hellmann (and ECD)

\subsection{Why it matters now!}
How long until this is a realistic threat?
As of now, there exists no known quantum computers powerful enough to factorize even trivially large numbers.
The current record on a quantum computer using Shor's algorithm is 21($3\times7$)\footnote{This is not the largest number factorized by quantum computers, that would be 56153. That was, however, using another algorithm, which is not fast for large numbers.}\cite{shor21}.

One important issue to consider is the security of today's communication from future attack.
An adversary could in theory record vast amounts of encrypted communication today, knowing that the information will be possible to decrypt when powerful quantum computers are available.
There are two conclusions to be drawn from this notion.
1) all communication that has been or is going to be encrypted with vulnerable schemes, must be regarded as theoretically compromised.
2) post-quantum encryption standards are not something that need only be in place before the introduction of powerful quantum computers, they are needed as soon as possible.
All information currently being encrypted using vulnerable schemes is ultimately vulnerable to a sufficiently resourceful or specifically interested adversary.


\bibliographystyle{IEEEtran}
\bibliography{sample}

\end{document}
