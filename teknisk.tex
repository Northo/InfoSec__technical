\documentclass[conference]{IEEEtran}
\IEEEoverridecommandlockouts
% The preceding line is only needed to identify funding in the first footnote. If that is unneeded, please comment it out.
\usepackage{cite}
\usepackage{amsmath,amssymb,amsfonts}
\usepackage{algorithmic}
\usepackage{graphicx}
\usepackage{textcomp}
\usepackage{xcolor}
\usepackage{quoting}
\def\BibTeX{{\rm B\kern-.05em{\sc i\kern-.025em b}\kern-.08em
    T\kern-.1667em\lower.7ex\hbox{E}\kern-.125emX}}
\begin{document}

\title{On post-quantum cryptography}

\author{\IEEEauthorblockN{Thorvald Ballestad}
\IEEEauthorblockA{\textit{Department of Physics} \\
\textit{NTNU}\\
Trondheim, Norway \\
thorvald.tb@gmail.com}}

\maketitle

\begin{abstract}
  The discovery of assymmetric encryption in the early 70s has blessed our society with secure and available encryption of online communication.
  In essence, the simple fact that the factorization problem appears to be very difficult to solve, allows us to use public-key encryption in the key agreement phase of network encryption.
  The advancement of real-world quantum computers threaten to destroy this privilege -- the factorization problem is simple on a quantum computer.
  There is only a matter of time before quantum computers will be powerful enough to break assymetric encryption -- which will destroy the very foundation of online security.
  To combat this, one seeks to introduce post-quantum cryptographic methods.
  Algorithms for encryption that can not simply be broken by quantum computers.
  In this paper, we will explore the need for new encryption in the era of quantum computing.
\end{abstract}

\begin{IEEEkeywords}
quantum, computer, encryption
\end{IEEEkeywords}

\section{Introduction}
At its core, all communication over the internet is essentially like shouting to each other in an crowded environment, open for anyone to hear.
At the infancy of the internet, this was fine, as the communication was only between trusted parties.
Today the situation is rather different, with personal details, photos, banking details, business secrets, and much more being transported over the internet.
The answer is the Transport Security Layer (TSL), which provides both authentication and encryption.
Authentication is the process of verifying the identity of the parties involved in the communication.
Encryption is the process of obfuscating the information in the communication, so that it is unreadable to anyone except the intended recipient(s).
There is an inherent issue with encryption over the web - how does the two parties, who have never met in person, establish a secure connection that is obfuscated for anyone else, when the establishment of such communication must happen over open communication?

\section{A quick introduction to encryption}
In the introduction we mentioned encryption and authentication, both of which are relevant in post-quantum cryptography.
For the sake of brevity however, we focus here mainly on encryption.
The answer to the problem of establishing a secure connection over open communication, when the involved parties has no shared secret before the process is of course asymmetric encryption.
The most well known asymmetric encryption algorithm is RSA, which the user might have heard of.
The Diffie-Hellman algorithm, with its many variants, is becoming increasingly more popular, for its many advantages over RSA.
They do both, however, rely on the same basic principle: asymmetric encryption are methods for establishing a secure connection between parties without requiring a pre-established shared secret.

Both RSA and Diffie-Hellman relies on the fact that there are no known algorithms for solving the hidden subgroup problem quickly.
The hidden subgroup problem is a generalization of the factorization problem.
As a practical example, consider the following written by the mathematician William Jevons in 1874:
\begin{quoting}
Can the reader say what two numbers multiplied together will produce the number 8616460799?[12] I think it unlikely that anyone but myself will ever know.
 
- William Stanley Jevons, 1874
\end{quoting}
The answer is $89681 \times 96079$, which is not difficult to find with modern computers.
However, the concept stands: given two primes a and b, it is simple to compute $N = a \times b$.
However, if a and b are sufficiently large, it is extremely hard to find a and b given only N.
A similar problem exists for discrete logarithms: find the integer $k$ such that $b^k = a$.
One may show that solving one of the problems, allows you to solve the other.

Both Diffie-Hellman and RSA are, though not terribly complicated, too lengthy to describe in depth here, so a quick outline will suffice.
Suppose we have two people, Alice and Bob, who which to establish encrypted communication.
Firstly, let us describe RSA.
Both Alice and Bob will have a pair of keys (numbers), one private and one public.
What this means is that the private key is a secret, while the public key is open for anyone to read.
The keys are such that something encrypted with one of the keys, may only be decrypted with the other.
One party, let us say Alice, will initiate the exchange by generating a new key, that will be the shared secret for the symmetric exchange -- this is the key that she must communicate to Bob, without anyone else being able to read it.
She then encrypts in with Bob's public key, and sends the result to Bob.
The only way to decrypt the message, is to use Bob's private key, which only Bob knows.
They now both have access to the symmetric key that Alice generated!
Diffie-Hellman works on a similar principle, but with slight variations.
Alice and Bob first generate number each, a and b, which are kept secret.
They also agree on some number g, which is open for anyone to see.
Alice then computes the number $g^a$ to Bob.
Bob takes that number, and calculates $(g^a)^b$.
He also computes $g^b$, which he sends to Alice.
By the simple fact that $(g^b)^a = g^{ab} = (g^b)^a$, they may now both compute $g^{ab}$, which is the symmetric key.
Notice that the only numbers ``available'' to anyone, the ones that are sent in the open, are $g, g^a and g^b$.
As was mentioned earlier, finding $a$ form $g^a$ is difficult, and thus an adversary is unable to find $g^{ba}$.
Of course, note also that $g^a g^b = g^{a+b}\neq g^{ab}$.


Note: the author feels the need to reiterate that we have here only discussed encryption, not authentication. Authentication is also of major importance to internet security.


For the sake of completeness, one should mention that asymmetric encryption is rarely used for the bulk of the communication.
The problem with asymmetric encryption is that it is slow, about 10'000 times slower than symmetric methods.
The solution is to use the asymmetric protocol to negotiate a symmetric key to be used for the rest of the communication.
This process of using asymmetric encryption to negotiate a symmetric key is known as \emph{key exchange}.


\section{Why are quantum computers a threat?}
It is this stage, the key exchange, that is vulnerable to quantum attacks.
The most used algorithms for key exchange today are dependent on the fact that the integer factorization problem is hard to solve (footnote: actually, the hidden subgroup problem for finite Abelian group).
As a practical exam, consider the following example written by the mathematician William Jevons in 1874:
\begin{quoting}
Can the reader say what two numbers multiplied together will produce the number 8616460799?[12] I think it unlikely that anyone but myself will ever know.
 
- William Stanley Jevons, 1874
\end{quoting}
Now, the answer is $89681 \times 96079$, which is not difficult to find with modern computers.
However, the concept stands: given two primes a and b, it is simple to compute $N = a \times b$.
However, if a and b are sufficiently large, it is extremely hard to find a and b given only N.
That is not the case for quantum computers, however.
Shor's algorithm can solve this problem in logarithmic time!
This threatens the integrity of modern encryption!

\subsection{When will the quantum computers be here?}
How long until this is a realistic threat?
As of now, there exists no quantum computers powerful enough to factorize even trivially large numbers.
The current record on a quantum computer using Shor's algorithm is 21 (3x7) (footnote: this is not the largest number factorized by quantum computers, that would be 56153. But that is using another algorithm, which is not fast for large numbers).

\subsection{The solution}
It is obvious that a post-quantum secure encryption standard has to rely on some different mechanism than the current standards -- that rely on the integer factorization problem.
As is common in cryptography research one tries to find methods that can be shown to reduce to known mathematical problems, for which there exist no known fast algorithm -- just like one did with the reduction of RSA and Diffie-Hellman into the hidden subgroup problem (integer factorization)).
In 2016 the National Institute of Science and Technology (NIST) initiated a standardization process for finding new methods of authentication an key exchange for the post-quantum era.
This process has gone through several rounds of discarding some submissions and moving on with the most promising.
As late as July 2020 the third round was announced, making this a very relevant topic.
In the third round seven submissions remain, four schemes for key exchange and three for authentication.

\subsection{Why it matters now!}
Another interesting issue to consider in this context is the security of today's communication from future attack.
An adversary could in theory record vast amounts of encrypted communication today, knowing that the information will be possible to decrypt when powerful quantum computers are available.
There are two conclusions to be drawn from this notion.
1) all communication that has been or is going to be encrypted with vulnerable schemes, must be regarded as theoretically compromised.
2) post-quantum encryption standards are not something that need only be in place before the introduction of powerful quantum computers, they are needed as soon as possible.
All information currently being encrypted using vulnerable schemes is ultimately vulnerable to a sufficiently resourceful or specifically interested adversary.


\section*{Acknowledgment}

The preferred spelling of the word ``acknowledgment'' in America is without 
an ``e'' after the ``g''. Avoid the stilted expression ``one of us (R. B. 
G.) thanks $\ldots$''. Instead, try ``R. B. G. thanks$\ldots$''. Put sponsor 
acknowledgments in the unnumbered footnote on the first page.

\section*{References}

Please number citations consecutively within brackets \cite{b1}. The 
sentence punctuation follows the bracket \cite{b2}. Refer simply to the reference 
number, as in \cite{b3}---do not use ``Ref. \cite{b3}'' or ``reference \cite{b3}'' except at 
the beginning of a sentence: ``Reference \cite{b3} was the first $\ldots$''

Number footnotes separately in superscripts. Place the actual footnote at 
the bottom of the column in which it was cited. Do not put footnotes in the 
abstract or reference list. Use letters for table footnotes.

Unless there are six authors or more give all authors' names; do not use 
``et al.''. Papers that have not been published, even if they have been 
submitted for publication, should be cited as ``unpublished'' \cite{b4}. Papers 
that have been accepted for publication should be cited as ``in press'' \cite{b5}. 
Capitalize only the first word in a paper title, except for proper nouns and 
element symbols.

For papers published in translation journals, please give the English 
citation first, followed by the original foreign-language citation \cite{b6}.

\begin{thebibliography}{00}
\bibitem{b1} G. Eason, B. Noble, and I. N. Sneddon, ``On certain integrals of Lipschitz-Hankel type involving products of Bessel functions,'' Phil. Trans. Roy. Soc. London, vol. A247, pp. 529--551, April 1955.
\bibitem{b2} J. Clerk Maxwell, A Treatise on Electricity and Magnetism, 3rd ed., vol. 2. Oxford: Clarendon, 1892, pp.68--73.
\bibitem{b3} I. S. Jacobs and C. P. Bean, ``Fine particles, thin films and exchange anisotropy,'' in Magnetism, vol. III, G. T. Rado and H. Suhl, Eds. New York: Academic, 1963, pp. 271--350.
\bibitem{b4} K. Elissa, ``Title of paper if known,'' unpublished.
\bibitem{b5} R. Nicole, ``Title of paper with only first word capitalized,'' J. Name Stand. Abbrev., in press.
\bibitem{b6} Y. Yorozu, M. Hirano, K. Oka, and Y. Tagawa, ``Electron spectroscopy studies on magneto-optical media and plastic substrate interface,'' IEEE Transl. J. Magn. Japan, vol. 2, pp. 740--741, August 1987 [Digests 9th Annual Conf. Magnetics Japan, p. 301, 1982].
\bibitem{b7} M. Young, The Technical Writer's Handbook. Mill Valley, CA: University Science, 1989.
\end{thebibliography}
\vspace{12pt}
\color{red}
IEEE conference templates contain guidance text for composing and formatting conference papers. Please ensure that all template text is removed from your conference paper prior to submission to the conference. Failure to remove the template text from your paper may result in your paper not being published.

\end{document}
